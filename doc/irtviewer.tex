\section{ロボットビューワ}

デフォルトのIrtviewerを作成する(\reffig{irtviewer}).
\begin{verbatim}
(make-irtviewer)
\end{verbatim}
\begin{figure}[htb]
  \begin{center}
    \includegraphics[width=0.50\columnwidth]{fig/irtviewer.jpg}
    \caption{Default irtviewer}
    \labfig{irtviewer}
  \end{center}
\end{figure}

Irtviewerを作成して,xy平面のグリッドを描画する(\reffig{irtviewer-floor}).
\begin{verbatim}
(make-irtviewer)
(send *irtviewer* :draw-floor 100)
(send *irtviewer* :draw-objects)
\end{verbatim}
\begin{figure}[htb]
  \begin{center}
    \includegraphics[width=0.50\columnwidth]{fig/irtviewer-floor.jpg}
    \caption{Irtviewer with floor grid}
    \labfig{irtviewer-floor}
  \end{center}
\end{figure}

Irtviewerを作成して,背景を白にして,xy平面のグリッドを黒で描画する(\reffig{irtviewer-floor-white}).
\begin{verbatim}
(make-irtviewer)
(send *irtviewer* :change-background (float-vector 1 1 1))
(send *irtviewer* :draw-floor 100)
(send *irtviewer* :floor-color #f(0 0 0))
(send *irtviewer* :draw-objects)
\end{verbatim}
\begin{figure}[htb]
  \begin{center}
    \includegraphics[width=0.50\columnwidth]{fig/irtviewer-floor-white.jpg}
    \caption{Irtviewer with white background and floor grid}
    \labfig{irtviewer-floor-white}
  \end{center}
\end{figure}


\input{irtviewer-func}


\subsection{Animation GIF}

\verb+with-save-animgif+ マクロを用いることでAnimation Gifファイルを生成することが出来る.


{\baselineskip=10pt
\begin{verbatim}
(with-save-animgif "file"
  (dotimes (a 45)
    (send *robot* :larm-shoulder-r :joint-angle (* a 4))
    (send *irtviewer*:draw-objects)))
\end{verbatim}
}

\verb+with-save-mpeg+を用いることで\verb+mpeg+ファイルの生成も可能である.

\subsection{Eps}

\verb+irtdraw-save+ を用いることでEpsファイルを生成することが出来る.

{\baselineskip=10pt
\begin{verbatim}
(defun irtdraw-demo-hand nil
  (load "irteus/demo/demo.l")
  (hand-grasp)
  (irtdraw-save :fname "hand-grasp.eps"
                :bodies (cons (cadr (objects))
                              (send (car (objects)) :bodies)))
  )
(defun irtdraw-dual-arm nil
  (load "irteus/demo/demo.l")
  (dual-arm-ik)
  (irtdraw-save :fname "dual-arm.eps"
                :bodies (flatten (send-all (objects) :bodies)))
  )
\end{verbatim}
}
